% Created 2020-12-10 Thu 21:44
% Intended LaTeX compiler: xelatex
\documentclass[12pt]{article}
\usepackage{graphicx}
\usepackage{grffile}
\usepackage{longtable}
\usepackage{wrapfig}
\usepackage{rotating}
\usepackage[normalem]{ulem}
\usepackage{amsmath}
\usepackage{textcomp}
\usepackage{amssymb}
\usepackage{capt-of}
\usepackage{hyperref}
\usepackage{minted}
\usepackage{amsmath}
\usepackage{amssymb}
\usepackage{setspace}
\usepackage{subcaption}
\usepackage{mathtools}
\usepackage{xfrac}
\usepackage[margin=1in]{geometry}
\usepackage[utf8]{inputenc}
\usepackage{color}
\usepackage{epsf}
\usepackage{tikz}
\usepackage{graphicx}
\usepackage{pslatex}
\usepackage{hyperref}
\usepackage[adobe-utopia]{mathdesign}
\usepackage{helvet}
\renewcommand{\familydefault}{\sfdefault}
\usepackage{textgreek}
\renewcommand*{\textgreekfontmap}{%
{phv/*/*}{LGR/neohellenic/*/*}%
{*/b/n}{LGR/artemisia/b/n}%
{*/bx/n}{LGR/artemisia/bx/n}%
{*/*/n}{LGR/artemisia/m/n}%
{*/b/it}{LGR/artemisia/b/it}%
{*/bx/it}{LGR/artemisia/bx/it}%
{*/*/it}{LGR/artemisia/m/it}%
{*/b/sl}{LGR/artemisia/b/sl}%
{*/bx/sl}{LGR/artemisia/bx/sl}%
{*/*/sl}{LGR/artemisia/m/sl}%
{*/*/sc}{LGR/artemisia/m/sc}%
{*/*/sco}{LGR/artemisia/m/sco}%
}
\makeatletter
\newcommand*{\rom}[1]{\expandafter\@slowromancap\romannumeral #1@}
\makeatother
\DeclarePairedDelimiterX{\infdivx}[2]{(}{)}{%
#1\;\delimsize\|\;#2%
}
\newcommand{\infdiv}{D\infdivx}
\DeclarePairedDelimiter{\norm}{\lVert}{\rVert}
\def\Z{\mathbb Z}
\def\R{\mathbb R}
\def\C{\mathbb C}
\def\N{\mathbb N}
\def\Q{\mathbb Q}
\def\noi{\noindent}
\onehalfspace
\author{Sandy Urazayev\thanks{University of Kansas (ctu@ku.edu)}}
\date{327; 12020 H.E.}
\title{Final Project Outline\\\medskip
\large MATH 596}
\hypersetup{
 pdfauthor={Sandy Urazayev},
 pdftitle={Final Project Outline},
 pdfkeywords={},
 pdfsubject={},
 pdfcreator={Emacs 28.0.50 (Org mode 9.3)}, 
 pdflang={English}}
\begin{document}

\maketitle
\section*{Introduction}
\label{sec:org5b6727a}
\begin{itemize}
\item Introduce the problem of data rotting and resolution sharpening
\item Introduce various examples of low-res images (old photos)
\end{itemize}

\section*{Data Overview}
\label{sec:orgee1b949}
\begin{itemize}
\item Talk about the data required for this application
\item Resolve around the quality of data needed for better results
\item Introduce the datasets we are using
\item Talk about data usage limitations
\end{itemize}

\section*{Design Implementation}
\label{sec:org5398d36}
\begin{itemize}
\item Talk about the convolution neural networks, how to design them
\item Discussion on input-output vector sizes
\item Talk about dividing data into training sets and validation sets
\item Discovering ways to record error from the ground-truth
\end{itemize}

\section*{Implementation}
\label{sec:org310c4b6}
\begin{itemize}
\item Actual step-by-step implementation of the problem
\item Include code snippets from Keras
\item Introduce annoying issues with neural networks and fine-tuning
\item More hardships encountered
\end{itemize}

\section*{Mathematical Basis}
\label{sec:org20a6c39}
\begin{itemize}
\item More in-depth look into the system
\item Talk about the math basis for the implemented layers
\item Backpropagation techniques when using convolution
\end{itemize}

\section*{Results}
\label{sec:org6c0b678}
\begin{itemize}
\item Show predictions for our test data
\item Record the errors from the ground truth data
\end{itemize}

\section*{Conclusion}
\label{sec:org520bb78}
\begin{itemize}
\item Conclusion on the problems
\item Areas for improvement
\item Comparing to cutting-edge research
\end{itemize}
\end{document}
